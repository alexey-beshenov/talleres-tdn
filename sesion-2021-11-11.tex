\ifdefined\handout
  \documentclass[handout]{beamer}
\else
  \documentclass{beamer}
\fi

\usetheme{boxes}
\definecolor{beamer@structure@color}{rgb}{0,0,0}

\usecolortheme{structure}

\usepackage{diagbox}

\usepackage{framed}

\setbeamertemplate{footline}[frame number]
\setbeamertemplate{frametitle}{\color{black}
\def\myhrulefill{\leavevmode\leaders\hrule height 1pt\hfill\kern 0pt}
\headingfont\insertframetitle\par\vskip-8pt\myhrulefill}

\usepackage[spanish]{babel}

\DeclareMathOperator{\mcd}{mcd}

\newcommand{\ZZ}{\mathbb{Z}}

\setbeamertemplate{navigation symbols}{}

\definecolor{shadecolor}{rgb}{0.89,0.89,0.70}

\usepackage{mathspec}
\setsansfont[BoldFont={IBM Plex Sans Bold}, ItalicFont={IBM Plex Sans Italic}]{IBM Plex Sans}
\setmathrm[BoldFont={IBM Plex Sans Bold}, ItalicFont={IBM Plex Sans Italic}]{IBM Plex Sans}
\newfontfamily\headingfont[]{IBM Plex Sans Bold}

\ifdefined\solutions
  \setbeamercovered{transparent=25}
\fi

\begin{document}

%%%%%%%%%%%%%%%%%%%%%%%%%%%%%%%%%%%%%%%%%%%%%%%%%%%%%%%%%%%%%%%%%%%%%%%%%%%%%%%%

\begin{frame}[noframenumbering]
  \begin{center}
    {\LARGE\bf En torno del

      teorema de dos cuadrados

      $(n = x^2 + y^2)$

    }

    \vspace{3em}

    {\large\bf Alexey Beshenov}

    \vspace{4em}

    11/11/2021

  \end{center}
\end{frame}

%%%%%%%%%%%%%%%%%%%%%%%%%%%%%%%%%%%%%%%%%%%%%%%%%%%%%%%%%%%%%%%%%%%%%%%%%%%%%%%%

\begin{frame}[fragile]
  \frametitle{Motivación}

  \begin{shaded}
    \textbf{IWYMIC 2004}: ¿Cuántas soluciones enteras tiene la ecuación
    $x^2 + y^2 - 16y = 2004$?
  \end{shaded}
\end{frame}

%%%%%%%%%%%%%%%%%%%%%%%%%%%%%%%%%%%%%%%%%%%%%%%%%%%%%%%%%%%%%%%%%%%%%%%%%%%%%%%%

\begin{frame}[fragile]
  \frametitle{La vez pasada}

  \begin{shaded}
    Para un primo impar $p$ la congruencia $x^2 \equiv -1 \pmod{p}$ tiene
    solución si y solo si $p \equiv 1 \pmod{4}$.
  \end{shaded}

  \onslide<2->{\begin{shaded}
    \textbf{Ejercicio}: Demuestre que $x^2 + y^2 \equiv 0 \pmod{p}$ tiene una
    solución $(x,y) \not\equiv (0,0) \pmod{p}$ si y solo si
    $p \equiv 1 \pmod{4}$.
  \end{shaded}}

  \vspace{\fill}

  \ifdefined\solutions
  \onslide<3->{\rule{2cm}{1pt}}

  \begin{itemize}
  \item<4-> Si $y \not\equiv 0$, entonces existe $y^{-1}$ tal que
    $y y^{-1} \equiv 1 \pmod{p}$.

  \item<5-> $x^2 + y^2 \equiv 0$, $y \not\equiv 0$ $\Longrightarrow$
    $(x\,y^{-1})^2 \equiv -1$ $\Longrightarrow$ $p \equiv 1 \pmod{4}$.

  \item<6-> $p \equiv 1 \pmod{4}$ $\Longrightarrow$ $\exists z$, $z^2 \equiv -1$.

    Para cualquier $x$, tenemos $x^2 + (\pm zx)^2 \equiv 0$.
  \end{itemize}
  \fi
\end{frame}

%%%%%%%%%%%%%%%%%%%%%%%%%%%%%%%%%%%%%%%%%%%%%%%%%%%%%%%%%%%%%%%%%%%%%%%%%%%%%%%%

\begin{frame}[fragile]
  \frametitle{Ejemplo numérico}

  \[ x^2 + y^2 \equiv 0 \pmod{p} \]

  \begin{itemize}
  \item<2-> $p = 5$.

  \item<3-> $(\pm 2)^2 \equiv 4 \equiv -1$.

  \item<4-> Ejemplo: $x = 2$, $y = \pm 2 x = \pm 4$.

  \item<5-> $(x,y) = (0,0), \, (1,2), \, (1,3), \, (2,1), \, (2,4), \, (3, 1), \, (3, 4), \, (4, 2), \, (4, 3)$.
  \end{itemize}
\end{frame}

%%%%%%%%%%%%%%%%%%%%%%%%%%%%%%%%%%%%%%%%%%%%%%%%%%%%%%%%%%%%%%%%%%%%%%%%%%%%%%%%

\begin{frame}[fragile]
  \frametitle{IWYMIC 2004}

  \begin{shaded}
    ¿Cuántas soluciones enteras tiene la ecuación $x^2 + y^2 - 16y = 2004$?
  \end{shaded}

  \vspace{\fill}

  \ifdefined\solutions
  \onslide<2->{\rule{2cm}{1pt}}

  \begin{itemize}
  \item<3-> $x^2 + (y - 8)^2 = 2068 = 4\cdot 11\cdot 47$.

    Pongamos $z = y-8$.

  \item<4-> $11 \equiv 47 \equiv 3 \pmod{4}$.

  \item<5-> $x^2 + z^2 \equiv 0 \pmod{11}$.

    $x \equiv z \equiv 0 \pmod{11}$ (¡no hay soluciones no triviales!)

  \item<6-> $11 \mid x$, $11 \mid z$ $\Longrightarrow$ $11^2 \mid (x^2 + z^2)$.

    Contradicción.

  \item<7-> Conclusión: ¡no hay soluciones!
  \end{itemize}
  \fi
\end{frame}

%%%%%%%%%%%%%%%%%%%%%%%%%%%%%%%%%%%%%%%%%%%%%%%%%%%%%%%%%%%%%%%%%%%%%%%%%%%%%%%%

\begin{frame}[fragile]
  \frametitle{Pequeña variación}

  \begin{shaded}
    \textbf{Tarea}: Verifique si $x^2 + y^2 - 16y = 2020$ tiene soluciones.
  \end{shaded}

  \vspace{1em}

  \ifdefined\solutions
  \onslide<2->{\rule{2cm}{1pt}}

  \begin{itemize}
  \item<3-> $x^2 + (y - 8)^2 = 2084 = 2^2 \cdot 521$.
 
  \item<4-> $521 \equiv 1 \pmod{4}$ y $521 = 11^2 + 20^2$.

  \item<5-> \textbf{Identidad de Diofanto} (ejercicio):

    $(a^2 + b^2)\,(c^2 + d^2) = (ac - bd)^2 + (ad + bc)^2$.

  \item<6-> $2084 = 2^2\cdot 521 = (0 + 2^2)\,(11^2 + 20^2) = 22^2 + 40^2$.

  \item<7-> $x = 22$, $y = 48$ es una solución.

  \item<8-> $8$ soluciones en total, que vinenen de $2084 = (\pm 22)^2 + (\pm 40)^2 = (\pm 40)^2 + (\pm 22)^2$.
\end{itemize}
\fi
\end{frame}

%%%%%%%%%%%%%%%%%%%%%%%%%%%%%%%%%%%%%%%%%%%%%%%%%%%%%%%%%%%%%%%%%%%%%%%%%%%%%%%%

\begin{frame}[fragile]
  \frametitle{Teorema de dos cuadrados}

  \begin{shaded}
    \textbf{Fermat} (1640): Un primo impar $p$ es una suma de dos cuadrados si y
    solamente si $p \equiv 1 \pmod{4}$.

    Primera prueba: Euler (1749).
  \end{shaded}

  \onslide<2->{\begin{align*}
    5 & = 1^2 + 2^2, & 13 & = 2^2 + 3^2, & 17 & = 1^2 + 4^2, \\
    29 & = 2^2 + 5^2, & 37 & = 1^2 + 6^2, & 41 & = 4^2 + 5^2, \\
    53 & = 2^2 + 7^2, & 61 & = 5^2 + 6^2, & 73 & = 3^2 + 8^2, \\
    89 & = 5^2 + 8^2, & 97 & = 4^2 + 9^2, & 101 & = 1^2 + 10^2, \\
    109 & = 3^2 + 10^2, & 113 & = 7^2 + 8^2, & 137 & = 4^2 + 11^2, \\
    149 & = 7^2 + 10^2, & 157 & = 6^2 + 11^2, & 173 & = 2^2 + 13^2.
  \end{align*}}
\end{frame}

%%%%%%%%%%%%%%%%%%%%%%%%%%%%%%%%%%%%%%%%%%%%%%%%%%%%%%%%%%%%%%%%%%%%%%%%%%%%%%%%

\begin{frame}[fragile]
  \frametitle{Prueba (Thue, 1902)}

  \onslide<2->{\textbf{Necesidad} (fácil):}

  \begin{itemize}
  \item<3-> $p = x^2 + y^2$ $\Longrightarrow$ $x^2 + y^2 \equiv 0 \pmod{p}$, $(x,y) \not\equiv (0,0)$

    $\Longrightarrow$ $p \equiv 1 \pmod{4}$.
  \end{itemize}

  \onslide<4->{\textbf{Suficiencia} (difícil):}

  \begin{itemize}
  \item<5-> $p \equiv 1 \pmod{4}$ $\Longrightarrow$ $z^2 \equiv -1$ tiene solución.

  \item<6-> $S = \{ (x', y') \in \ZZ^2 \mid 0 \le x', y' \le \sqrt{p} \}$.

  \item<7-> $|S| = (\lfloor \sqrt{p}\rfloor + 1)^2 > p$.

    Principio de las casillas: $(x',y') \ne (x'', y'')$ tales que
    \[ x' - zy' \equiv x'' - zy'' \pmod{p}
      \iff x' - x'' \equiv z\,(y' - y''). \]

  \item<8-> $x = |x' - x''|$, $y = |y' - y''|$.

  \item<9-> $x^2 + y^2 \equiv 0 \pmod{p}$, $0 < x^2 + y^2 < 2p$,

    $\Longrightarrow$ $x^2 + y^2 = p$. \qed
  \end{itemize}
\end{frame}

%%%%%%%%%%%%%%%%%%%%%%%%%%%%%%%%%%%%%%%%%%%%%%%%%%%%%%%%%%%%%%%%%%%%%%%%%%%%%%%%

\begin{frame}[fragile]
  \frametitle{Ejemplo numérico}

  \begin{itemize}
  \item<2-> $p = 29$, $\lfloor \sqrt{p}\rfloor = 5$.

  \item<3-> $12^2 \equiv -1 \pmod{p}$.

  \item<4-> Consideremos $x' - 12\,y' \pmod{p}$, $0 < x', y' \le 5$.

    \begin{center}
      \begin{tabular}{|c|cccccc|}
        \hline
        \backslashbox{$x'$}{$y'$} & $0$ & $1$ & $2$ & $3$ & $4$ & $5$ \\
        \hline
        $0$ & $0$ & $17$ & $5$ & $22$ & $10$ & $27$ \\
        $1$ & \boxed{$1$} & $18$ & $6$ & $23$ & $11$ & $28$ \\ 
        $2$ & $2$ & $19$ & $7$ & $24$ & $12$ & $0$ \\
        $3$ & $3$ & $20$ & $8$ & $25$ & $13$ & \boxed{$1$} \\
        $4$ & $4$ & $21$ & $9$ & $26$ & $14$ & $2$ \\ 
        $5$ & $5$ & $22$ & $10$ & $27$ & $15$ & $3$ \\
        \hline
      \end{tabular}
    \end{center}

  \item<5-> $1 - 12\cdot 0 \equiv 3 - 12\cdot 5$ $\iff$ $1-3 \equiv 12\cdot (0-5)$.

  \item<6-> $(1-3)^2 + (0-5)^2 \equiv 0$.

  \item<7-> $29 = 2^2 + 5^2$.
  \end{itemize}
\end{frame}

%%%%%%%%%%%%%%%%%%%%%%%%%%%%%%%%%%%%%%%%%%%%%%%%%%%%%%%%%%%%%%%%%%%%%%%%%%%%%%%%

\begin{frame}[fragile]
  \frametitle{Números compuestos}

  \begin{shaded}
    $n = p_1^{e_1} \cdots p_s^{e_s}$ es una suma de dos cuadrados si y solo si
    cada $p_i \equiv 3 \pmod{4}$ tiene potencia par $e_i$.
  \end{shaded}

  \onslide<2->{\textbf{Suficiencia}:}

  \begin{itemize}
  \item<3-> $1 = 1^2 + 0^2$, $2 = 1^2 + 1^2$.

  \item<4-> Si $p \equiv 1 \pmod{4}$, entonces $p = x^2 + y^2$.

  \item<5-> Si $m = a^2 + b^2$, $n = c^2 + d^2$,

    $mn = (ac - bd)^2 + (ad + bc)^2$.

  \item<6-> Si $m = a^2 + b^2$, entonces $mn^2 = (an)^2 + (bn)^2$.
  \end{itemize}

  \onslide<7->{\textbf{Necesidad}:}

  \begin{itemize}
  \item<8-> $n = x^2 + y^2$, $p_i \mid n$, $p_i \equiv 3 \pmod{4}$.

  \item<9-> $p_i \mid x$, $p_i \mid y$ $\Longrightarrow$ $p_i^2 \mid n$,
    $n/p_i^2 = (x/p_i)^2 + (y/p_i)^2$.

  \item<10-> $n/p_i^2 = p_1^{e_1} \cdots p_i^{e_i - 2} \cdots p_s^{e_s}$.

  \item<11-> Si $p_i \equiv 3 \pmod{4}$, entonces $p_i \ne \square + \square$.
  \end{itemize}
\end{frame}

%%%%%%%%%%%%%%%%%%%%%%%%%%%%%%%%%%%%%%%%%%%%%%%%%%%%%%%%%%%%%%%%%%%%%%%%%%%%%%%%

\begin{frame}[fragile]
  \frametitle{Ejercicio}
  \begin{shaded}
    Verifique cuáles de los siguientes números tienen forma $n = x^2 + y^2$.
    Encuentre $x$ e $y$ correspondiente.
    \[ n = 160, ~ 208, ~ 230, ~ 351, ~ 585, ~ 715. \]
  \end{shaded}

  \ifdefined\solutions
  \onslide<2->{\rule{2cm}{1pt}}

  \begin{align*}
    \onslide<3->{160} & \onslide<3->{= 2^5\cdot 5,} & \quad \onslide<3->{160} & \onslide<3->{= 4^2 + 12^2;} \\
    \onslide<4->{208} & \onslide<4->{= 2^4\cdot 13,} & \quad \onslide<4->{208} & \onslide<4->{= 8^2 + 12^2;} \\
    \onslide<5->{230} & \onslide<5->{= 2\cdot 5\cdot \boxed{23},} & \quad \onslide<5->{230} & \onslide<5->{\ne \square + \square;} \\
    \onslide<6->{351} & \onslide<6->{= \boxed{3^3}\cdot 13,} & \quad \onslide<6->{351} & \onslide<6->{\ne \square + \square;} \\
    \onslide<7->{585} & \onslide<7->{= 3^2\cdot 5\cdot 13,} & \quad \onslide<7->{585} & \onslide<7->{= 3^2 + 24^2 = 12^2 + 21^2;} \\
    \onslide<8->{715} & \onslide<8->{= 5\cdot \boxed{11}\cdot 13,} & \quad \onslide<8->{715} & \onslide<8->{\ne \square + \square.}
  \end{align*}
  \fi
\end{frame}

%%%%%%%%%%%%%%%%%%%%%%%%%%%%%%%%%%%%%%%%%%%%%%%%%%%%%%%%%%%%%%%%%%%%%%%%%%%%%%%%

\begin{frame}[fragile]
  \frametitle{¿De dónde viene la identidad de Diofanto?}

  \[ \tag{*} (a^2 + b^2)\,(c^2 + d^2) = (ac - bd)^2 + (ad + bc)^2. \]

  \begin{itemize}
  \item<2-> De la «Aritmética» de Diofanto ($\sim$ 250 d.C.).

  \item<3-> ¡También de los números complejos!

  \item<4-> \textbf{Números complejos}: $z = x + yi$, donde $i^2 = -1$.

  \item<5-> \textbf{El conjugado}: $\overline{z} = x - yi$.

    \textbf{La norma}: $N (z) = z\,\overline{z}$.

  \item<6-> \textbf{Ejercicio}:

    \begin{enumerate}
    \item $z\,\overline{z} = x^2 + y^2$,
    \item $\overline{zw} = \overline{z}\cdot \overline{w}$,
    \item $N (zw) = N(z)\,N(w)$.
    \end{enumerate}

  \item<7-> $(a + bi)\,(c + di) = (ac - bd) + (ad + bc)\,i$.

    Tomando $N (\cdots)$, se recupera (*).
  \end{itemize}
\end{frame}

%%%%%%%%%%%%%%%%%%%%%%%%%%%%%%%%%%%%%%%%%%%%%%%%%%%%%%%%%%%%%%%%%%%%%%%%%%%%%%%%

\begin{frame}[fragile]
  \frametitle{Número de representaciones $n = x^2 + y^2$}

  \begin{shaded}
    \[ C(n) = \# \{ (x,y) \in \ZZ^2 \mid x^2 + y^2 = n, ~ x > 0, y \ge 0 \} \]

    \[ n = p_1^{e_1} \cdots p_s^{e_s}, \quad e_i \text{ par si }p_i \equiv 3~(4) \]

    \[ C(n) = \prod_{p_i \equiv 1 ~ (4)} (e_i + 1) \]
  \end{shaded}

  \onslide<2->{Ejemplo:
  \begin{itemize}
  \item $n = 325$,

  \item $n = 5^2 \cdot 13$,

  \item $C(n) = 6$,

  \item $(x,y) = (1, 18), ~ (6, 17), ~ (10, 15), ~ (15, 10), ~ (17, 6), ~ (18, 1)$.
  \end{itemize}}
\end{frame}

%%%%%%%%%%%%%%%%%%%%%%%%%%%%%%%%%%%%%%%%%%%%%%%%%%%%%%%%%%%%%%%%%%%%%%%%%%%%%%%%

\begin{frame}[fragile]
  \frametitle{Tarea}

  \begin{shaded}
    ¿Cuántas soluciones enteras tiene $x^2 + y^2 - 16y = 1956$?
  \end{shaded}
\end{frame}

%%%%%%%%%%%%%%%%%%%%%%%%%%%%%%%%%%%%%%%%%%%%%%%%%%%%%%%%%%%%%%%%%%%%%%%%%%%%%%%%

\begin{frame}[fragile]
  \frametitle{Raíces de la unidad complejas y mód p}

  \begin{itemize}
  \item<2-> $i^2 = -1$, $i^3 = -i$, $i^4 = 1$.

    $i$ es una \textbf{cuarta raíz de la unidad}.
  \end{itemize}

  \onslide<3->{\begin{shaded}
      \textbf{Ejercicio}:
      $z = -\frac{1}{2} + \frac{\sqrt{3}}{2}\,i$ es una
      \textbf{raíz cúbica de la unidad}.

      Es decir, $z \ne 1$, $z^2 \ne 1$, $z^3 = 1$.
  \end{shaded}}

  \begin{itemize}
  \item<4-> Analogía con residuos mód $p$:

    $z^2 \equiv -1 \pmod{p}$ tiene solución si y solo si $p \equiv 1 \pmod{4}$.

    $z$ es una «cuarta raíz de la unidad mód $p$».
  \end{itemize}

  \onslide<5->{\begin{shaded}
    \textbf{Ejercicio}: una raíz cúbica de la unidad mód $p$ existe si y solo si
    $p \equiv 1 \pmod{3}$.

    (Es decir, $z \not\equiv 1$, $z^2 \not\equiv 1$, $z^3 \equiv 1$.)
  \end{shaded}}

  \onslide<6->{Ejemplo:\\
  $2^3 \equiv 1 \pmod{7}$, $3^3 \equiv 1 \pmod{13}$, $7^3 \equiv 1 \pmod{19}$,
  etc.}
\end{frame}

%%%%%%%%%%%%%%%%%%%%%%%%%%%%%%%%%%%%%%%%%%%%%%%%%%%%%%%%%%%%%%%%%%%%%%%%%%%%%%%%

\begin{frame}[fragile]
  \begin{center}
    \LARGE\bf ¡Gracias!
  \end{center}
\end{frame}

%%%%%%%%%%%%%%%%%%%%%%%%%%%%%%%%%%%%%%%%%%%%%%%%%%%%%%%%%%%%%%%%%%%%%%%%%%%%%%%%

\end{document}
