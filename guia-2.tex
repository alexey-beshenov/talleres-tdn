\documentclass{article}

\usepackage{fullpage}

\usepackage[utf8]{inputenc}
\usepackage[spanish]{babel}

\usepackage[T1]{fontenc}
\usepackage{fourier}
\usepackage{paratype}

\usepackage{framed}
\usepackage{hyperref}
\usepackage{amsmath,amssymb}
\usepackage{tikz}

\newcommand{\ZZ}{\mathbb{Z}}
\DeclareMathOperator{\mcd}{mcd}
\DeclareMathOperator{\mcm}{mcm}
\DeclareMathOperator{\ord}{ord}

\usepackage{amsthm}
\theoremstyle{definition}
\newtheorem{problema}{Problema}[section]
\newtheorem*{comentario}{Comentario}
\newtheorem*{definicion}{Definición}
\newtheorem*{ejemplo}{Ejemplo}

\title{Algunos ejercicios de la teoría de números elemental (continuación)}
\author{Alexey Beshenov (cadadr@gmail.com)}
\date{23 de septiembre de 2021}

\newenvironment{solucion}{\begin{proof}[Solución]\small}{\end{proof}}

\begin{document}

\maketitle

%%%%%%%%%%%%%%%%%%%%%%%%%%%%%%%%%%%%%%%%%%%%%%%%%%%%%%%%%%%%%%%%%%%%%%%%%%%%%%%%

Este texto es esencialmente una larga lista de problemas.

\setcounter{section}{-1}
\section{Restos mód $n$ (breve repaso)}

\begin{definicion}
  Fijemos algún $n = 1,2,3,4,\ldots$ Consideremos la siguiente relación sobre
  los números enteros: se dice que $a$ y $b$ son \textbf{congruentes módulo $n$}
  si $n$ divide a $a-b$:
  $$a \equiv b \pmod{n} \iff n \mid (a-b).$$
  En otras palabras, $a$ y $b$ tienen el mismo resto de la división por $n$.
\end{definicion}

\begin{problema}
  Demuestre que la congruencia mód $n$ es una relación de equivalencia sobre
  $\ZZ$; es decir, para cualesquiera $a,b,c\in \ZZ$
  \[
    a\equiv a, \quad
    a\equiv b \Rightarrow b\equiv a, \quad
    a\equiv b \text{ y } b\equiv c \Rightarrow a\equiv c.
  \]
\end{problema}

\begin{definicion}
  Las clases de equivalencia se llaman los \textbf{restos módulo $n$}. La clase
  de equivalencia de $a$ será denotada por $[a]_n$, o simplemente por $[a]$.
  El conjunto de los restos mód $n$ se denota por $\ZZ/n\ZZ$. Este tiene
  precisamente $n$ elementos:
  $$\ZZ/n\ZZ = \{ [0], \, [1], \, \ldots, \, [n-1] \}.$$
\end{definicion}

\begin{problema}
  Demuestre que si $a \equiv a'$, $b \equiv b'$, entonces
  \[ a+b \equiv a'+b', \quad a\cdot b \equiv a'\cdot b'. \]

  Esto quiere decir que la adición y multiplicación tiene sentido para los
  restos mód $n$: podemos definir
  \[ [a] + [b] = [a+b], \quad [a]\cdot [b] = [a\cdot b]. \]
\end{problema}

En lugar de $[0]_n$ y $[1]_n$ será conveniente escribir simplemente $0$ y $1$.

\begin{ejemplo}
  He aquí las tablas de adición y multiplicación módulo $5$:
  \[
    \begin{array}{c|ccccc}
      + & 0 & 1 & 2 & 3 & 4 \\
      \hline
      0 & 0 & 1 & 2 & 3 & 4 \\
      1 & 1 & 2 & 3 & 4 & 0 \\
      2 & 2 & 3 & 4 & 0 & 1 \\
      3 & 3 & 4 & 0 & 1 & 2 \\
      4 & 4 & 0 & 1 & 2 & 3
    \end{array}
    \quad\quad
    \begin{array}{c|ccccc}
      \cdot & 0 & 1 & 2 & 3 & 4 \\
      \hline
      0 & 0 & 0 & 0 & 0 & 0 \\
      1 & 0 & 1 & 2 & 3 & 4 \\
      2 & 0 & 2 & 4 & 1 & 3 \\
      3 & 0 & 3 & 1 & 4 & 2 \\
      4 & 0 & 4 & 3 & 2 & 1
    \end{array}
  \]
\end{ejemplo}

\begin{ejemplo}
  He aquí las tablas de adición y multiplicación módulo $6$:
  \[
    \begin{array}{c|cccccc}
      + & 0 & 1 & 2 & 3 & 4 & 5 \\
      \hline
      0 & 0 & 1 & 2 & 3 & 4 & 5 \\
      1 & 1 & 2 & 3 & 4 & 5 & 0 \\
      2 & 2 & 3 & 4 & 5 & 0 & 1 \\
      3 & 3 & 4 & 5 & 0 & 1 & 2 \\
      4 & 4 & 5 & 0 & 1 & 2 & 3 \\
      5 & 5 & 0 & 1 & 2 & 3 & 4
    \end{array}
    \quad\quad
    \begin{array}{c|cccccc}
      \cdot & 0 & 1 & 2 & 3 & 4 & 5 \\
      \hline
      0 & 0 & 0 & 0 & 0 & 0 & 0 \\
      1 & 0 & 1 & 2 & 3 & 4 & 5 \\
      2 & 0 & 2 & 4 & 0 & 2 & 4 \\
      3 & 0 & 3 & 0 & 3 & 0 & 3 \\
      4 & 0 & 4 & 2 & 0 & 4 & 2 \\
      5 & 0 & 5 & 4 & 3 & 2 & 1
    \end{array}
  \]
\end{ejemplo}

\begin{problema}
  Demuestre que las ecuaciones
  $$3x^2 + 2 = y^2, \quad 7x^3 + 2 = y^3$$
  no tienen soluciones $x,y \in \ZZ$, usando reducción módulo algunos $p$.
\end{problema}

\begin{problema}
  \label{probl:pequeno-Fermat}
  Sea $p$ un número primo.

  \begin{enumerate}
  \item[a)] Demuestre que $p\mid {p\choose k}$ para $k = 1,2,\ldots,p-1$.

  \item[b)] Deduzca de a) que ${p-1\choose k} \equiv (-1)^k \pmod{p}$.

  \item[c)] Deduzca de a) la <<fórmula del binomio mód $p$>>:
    $$(a+b)^p \equiv a^p + b^p \pmod{p}$$
    para cualesquiera $a,b \in \ZZ$.

  \item[d)] Deduzca de c) el \textbf{pequeño teorema de Fermat}:
    $a^p \equiv a \pmod{p}$ para todo $a \in \ZZ$.

    Sugerencia: use inducción con caso base $a = 0$ y el paso inductivo mediante
    $(a+1)^p \equiv a^p + 1 \pmod{p}$.
  \end{enumerate}
\end{problema}

\begin{problema}
  Demuestre las siguientes congruencias mód $p$:
  \begin{align*}
    1 + 2 + 3 + \cdots + (p-1) & \equiv 0 \text{ para }p \ne 2,\\
    1^2 + 2^2 + 3^2 + \cdots + (p-1)^2 & \equiv 0 \text{ para }p \ne 2,3,\\
    1^3 + 2^3 + 3^3 + \cdots + (p-1)^3 & \equiv 0 \text{ para }p \ne 2.
  \end{align*}
\end{problema}

\section{Restos invertibles mód $n$}

\begin{definicion}
  Se dice que un resto $[a]_n$ es \textbf{invertible} si existe otro resto
  $[b]_n$ tal que $[a]_n \cdot [b]_n = 1$
  (o de manera equivalente, $ab \equiv 1 \pmod{n}$).
  También se escribe $[b]_n = [a]_n^{-1}$.
\end{definicion}

El conjunto de los restos invertibles módulo $n$ se denotará por
$(\ZZ/n\ZZ)^\times$.

\begin{ejemplo}
  He aquí los restos módulo $n = 15$ y sus inversos; <<--->> significa que el
  resto no es invertible.

  \begin{center}
    \begin{tabular}{ccccccccccccccc}
      \hline
      $0$ & $1$ & $2$ & $3$ & $4$ & $5$ & $6$ & $7$ & $8$ & $9$ & $10$ & $11$ & $12$ & $13$ & $14$ \\
      \hline
      --- & $1$ & $8$ & --- & $4$ & --- & --- & $13$ & $2$ & --- & --- & $11$ & --- & $7$ & $14$ \\
      \hline
    \end{tabular}
  \end{center}
\end{ejemplo}

\begin{problema}[Cancelación]
  Demuestre que si $x,y,z$ son restos módulo $n$, y $z$ es invertible,
  entonces $xz = yz$ implica que $x = y$.
  ¿Qué pasa si $z$ no es invertible?
\end{problema}

\begin{problema}
  \label{probl:invertible-coprimo}
  En este problema vamos a probar que $[a]_n$ es invertible si y solo si
  $\mcd (a,n) = 1$:
  \[ (\ZZ/n\ZZ)^\times = \{ [a]_n \mid 0 \le a < n, ~ \mcd (a,n) = 1 \}. \]

  \begin{enumerate}
  \item[a)] Si $\mcd (a,n) = 1$, use la identidad de Bézout para encontrar
    $[a]_n^{-1}$.

  \item[b)] Demuestre que si $x,y \in \ZZ/n\ZZ$ son dos restos no nulos tales
    que $xy = 0$, entonces $x$ e $y$ no pueden ser invertibles.

    (En otras palabras, si $n \mid ab$ para $n \nmid a$, $n \nmid b$, entonces
    $a$ e $b$ no son invertibles módulo $n$.)

  \item[c)] Si $\mcd (a,n) > 1$, use el punto anterior para probar que $a$ no es
    invertible mód $n$.
  \end{enumerate}
\end{problema}

\begin{problema}
  Para $n = 5, 6, 8$ encuentre cuáles restos mód $n$ son invertibles y
  escriba sus inversos correspondientes.
\end{problema}

\begin{problema}
  Calcule $[6]_{385}^{-1}$.
\end{problema}

\begin{problema}[Teorema de Wilson]
  \label{probl:Wilson-1}
  Sea $p$ un primo.

  \begin{enumerate}
  \item[a)] Demuestre que para todo $x \in (\ZZ/p\ZZ)^\times$ se tiene
    $x^{-1} = x$ si y solamente si $x = \pm 1$.

  \item[b)] Deduzca de a) que $(p - 1)! \equiv -1 \pmod{p}$.

  \item[c)] Demuestre que $(n-1)! \equiv 0 \pmod{n}$ si $n \ge 6$ es compuesto.
  \end{enumerate}
\end{problema}

\begin{problema}
  Demuestre que para todo primo $p$, el numerador de
  $1 + \frac{1}{2} + \frac{1}{3} + \cdots + \frac{1}{p-1}$
  es divisible por $p$.

  Sugerencia: $1, 2, \ldots, p-1$ son invertibles mód $p$.
\end{problema}

%%%%%%%%%%%%%%%%%%%%%%%%%%%%%%%%%%%%%%%%%%%%%%%%%%%%%%%%%%%%%%%%%%%%%%%%%%%%%%%%

\section{Teorema chino del resto}

\begin{problema}[Teorema chino del resto]
  Sean $m$ y $n$ dos enteros coprimos ($\mcd (m,n) = 1$).

  \begin{itemize}
  \item[a)] Demuestre que existen enteros $e_1$, $e_2$ tales que
    \begin{gather*}
      e_1 \equiv 1 \pmod{m}, \quad e_1 \equiv 0 \pmod{n}, \\
      e_2 \equiv 0 \pmod{m}, \quad e_2 \equiv 1 \pmod{n}.
    \end{gather*}
    Sugerencia: identidad de Bézout.

  \item[b)] Dados $a, b \in \ZZ$, encuentre $c$ tal que
    $$c \equiv a \pmod{m}, \quad c \equiv b \pmod{n}.$$
  \end{itemize}
\end{problema}

Podemos reformular el resultado de arriba de la siguiente manera.

\begin{framed}
  Para $\mcd (m,n) = 1$ la aplicación
  \begin{align*}
    \Phi\colon \ZZ/mn\ZZ & \to \ZZ/m\ZZ \times \ZZ/n\ZZ,\\
    [c]_{mn} & \mapsto ([c]_m, \, [c]_n)
  \end{align*}
  es biyectiva.
\end{framed}

\begin{proof}
  Acabamos de probar, que para todo par de restos
  $([a]_m, [b]_n)$ habrá $[c]_{mn}$ tal que $\Phi ([c]_{mm}) = ([a]_m,[b]_n)$.
  Además, usando que
  $$|\ZZ/mn\ZZ| = |\ZZ/m\ZZ \times \ZZ/n\ZZ| = |\ZZ/m\ZZ| \times |\ZZ/n\ZZ|,$$
  podemos concluir que $\Phi$ es también inyectiva.
\end{proof}

\begin{ejemplo}
  He aquí la aplicación $\Phi$ para $(m,n) = (2,3)$:
  \begin{gather*}
    \Phi\colon \ZZ/6\ZZ \to \ZZ/2\ZZ \times \ZZ/3\ZZ,\\
    0 \mapsto (0,0), ~
    1 \mapsto (1,1), ~
    2 \mapsto (0,2), ~
    3 \mapsto (1,0), ~
    4 \mapsto (0,1), ~
    5 \mapsto (1,2).
  \end{gather*}
\end{ejemplo}

\begin{problema}
  \label{probl:crt-invertibles}
  Demuestre que la biyección
  $$\Phi\colon \ZZ/mn\ZZ \to \ZZ/m\ZZ \times \ZZ/n\ZZ$$
  se restringe a una biyección entre los elementos invertibles
  \begin{align*}
    (\ZZ/mn\ZZ)^\times & \to (\ZZ/m\ZZ)^\times \times (\ZZ/n\ZZ)^\times, \\
    [a]_{mn} & \mapsto ([a]_m, [a]_n).
  \end{align*}
\end{problema}

\begin{problema}
  Formule y demuestre una versión del teorema chino del resto para módulos
  $n_1, \ldots, n_s$ tales que $\mcd (n_i,n_j) = 1$ para $i \ne j$.
\end{problema}

\begin{problema}
  Sea $f (x) = a_m\,x^m + \cdots + a_1\,x + a_0$ un polinomio con coeficientes
  enteros.

  \begin{enumerate}
  \item[a)] Demuestre que la congruencia $f (x) \equiv 0 \pmod{n}$ tiene
    solución para $n = p_1^{e_1} \cdots p_s^{e_s}$ si y solamente si
    $f (x) \equiv 0 \pmod{p_i^{e_i}}$ tiene solución para cada $i = 1,\ldots,s$.

  \item[b)] Sea $N$ el número de soluciones de $f (x) \equiv 0 \pmod{n}$ y $N_i$
    el número de soluciones de $f (x) \equiv 0 \pmod{p_i^{e_i}}$. Demuestre que
    $N = N_1 \cdots N_s$.
  \end{enumerate}
\end{problema}

\begin{problema}
  ~

  \begin{enumerate}
  \item[a)] Demuestre que $x^2 \equiv x \pmod{p^e}$ tiene únicas soluciones
    $x = 0$ y $1$ para todo primo $p$ y $e = 1,2,3,\ldots$

  \item[b)] En general, ¿cuántas soluciones tiene $x^2 \equiv x \pmod{n}$?
  \end{enumerate}
\end{problema}

\begin{problema}
  ~

  \begin{enumerate}
  \item[a)] Demuestre que $x^2 \equiv 1 \pmod{p^e}$ tiene únicas soluciones
    $x = \pm 1$ para $p$ impar y $e = 1,2,3,\ldots$

  \item[b)] Demuestre que $x^2 \equiv 1 \pmod{2^e}$ tiene $4$ soluciones para
    $e \ge 3$. ¿Cuáles son?

  \item[c)] En general, ¿cuántas soluciones tiene $x^2 \equiv 1 \pmod{n}$?
  \end{enumerate}
\end{problema}

\begin{problema}
  Use el teorema chino del resto para encontrar las soluciones de
  $x^2 \equiv x$ y $x^2 \equiv 1$ mód $n = 221$.
\end{problema}

%%%%%%%%%%%%%%%%%%%%%%%%%%%%%%%%%%%%%%%%%%%%%%%%%%%%%%%%%%%%%%%%%%%%%%%%%%%%%%%%

\section{Función $\phi$ de Euler}

\begin{definicion}
  Para un número natural $n$, la función de Euler es el número de los restos
  invertibles mód $n$:
  $$\phi (n) = \# (\ZZ/n\ZZ)^\times.$$
  O de manera equivalente (problema~\ref{probl:invertible-coprimo}),
  es el número de $1 \le a < n$ coprimos con $n$:
  $$\phi (n) = \# \{ 1 \le a < n \mid \mcd (a,n) = 1 \}.$$
\end{definicion}

\begin{ejemplo}
  He aquí algunos valores de $\phi (n)$:

  \begin{center}
    \begin{tabular}{rcccccccccccccccccccc}
      \hline
      $n\colon$ & $1$& $2$& $3$& $4$& $5$& $6$& $7$& $8$& $9$& $10$& $11$& $12$& $13$& $14$& $15$& $16$& $17$& $18$& $19$& $20$ \\
      $\phi(n)\colon$ & $1$& $1$& $2$& $2$& $4$& $2$& $6$& $4$& $6$& $4$& $10$& $4$& $12$& $6$& $8$& $8$& $16$& $6$& $18$& $8$ \\
      \hline
      $n\colon$ & $21$& $22$& $23$& $24$& $25$& $26$& $27$& $28$& $29$& $30$& $31$& $32$& $33$& $34$& $35$& $36$& $37$& $38$& $39$& $40$ \\
      $\phi(n)\colon$ & $12$& $10$& $22$& $8$& $20$& $12$& $18$& $12$& $28$& $8$& $30$& $16$& $20$& $16$& $24$& $12$& $36$& $18$& $24$& $16$ \\
      \hline
      $n\colon$ & $41$& $42$& $43$& $44$& $45$& $46$& $47$& $48$& $49$& $50$& $51$& $52$& $53$& $54$& $55$& $56$& $57$& $58$& $59$& $60$ \\
      $\phi(n)\colon$ & $40$& $12$& $42$& $20$& $24$& $22$& $46$& $16$& $42$& $20$& $32$& $24$& $52$& $18$& $40$& $24$& $36$& $28$& $58$& $16$ \\
      \hline
      $n\colon$ & $61$& $62$& $63$& $64$& $65$& $66$& $67$& $68$& $69$& $70$& $71$& $72$& $73$& $74$& $75$& $76$& $77$& $78$& $79$& $80$ \\
      $\phi(n)\colon$ & $60$& $30$& $36$& $32$& $48$& $20$& $66$& $32$& $44$& $24$& $70$& $24$& $72$& $36$& $40$& $36$& $60$& $24$& $78$& $32$ \\
      \hline
      $n\colon$ & $81$& $82$& $83$& $84$& $85$& $86$& $87$& $88$& $89$& $90$& $91$& $92$& $93$& $94$& $95$& $96$& $97$& $98$& $99$& $100$ \\
      $\phi(n)\colon$ & $54$& $40$& $82$& $24$& $64$& $42$& $56$& $40$& $88$& $24$& $72$& $44$& $60$& $46$& $72$& $32$& $96$& $42$& $60$& $40$ \\
      \hline
    \end{tabular}
  \end{center}
\end{ejemplo}

\begin{problema}
  \label{probl:phi-para-primario}
  Calcule que para un primo $p$ y $e = 1,2,3,\ldots$ se tiene
  $$\phi (p^e) = p^e - p^{e-1} = p^e\,\left(1 - \frac{1}{p}\right).$$
\end{problema}

\begin{problema}[Multiplicatividad]
  \label{probl:phi-multiplicativo}
  Demuestre que si $\mcd (m,n) = 1$, entonces
  $$\phi (mn) = \phi(m)\,\phi(n).$$

  Sugerencia: véase el problema \ref{probl:crt-invertibles}.
\end{problema}

\begin{problema}
  Deduzca de \ref{probl:phi-para-primario} y \ref{probl:phi-multiplicativo} la
  fórmula
  $$\phi (n) = n\,\prod_{p \mid n} \left(1 - \frac{1}{p}\right),$$
  donde el producto es sobre todos los divisores primos de $n$.
\end{problema}

\begin{problema}
  ~

  \begin{itemize}
  \item[a)] Demuestre que $\phi(n)$ es par para $n \ge 3$.

  \item[b)] ¿Para cuáles $n$ se tiene $\phi (n) \le 10$?

  \item[c)] ¿Para cuáles $n$ se tiene $\phi (n) = 100$?
  \end{itemize}
\end{problema}

\begin{problema}
  Demuestre que $\phi (n) \le n-1$ para $n \ge 2$, y la igualdad se cumple si y
  solo si $n = p$ es primo.
\end{problema}

\begin{problema}[Congruencia de Euler]
  \label{probl:congruencia-de-euler}
  Sean
  \[ (\ZZ/n\ZZ)^\times = \{ x_1, x_2, \ldots, x_{\phi (n)} \} \]
  los restos invertibles mód $n$.

  \begin{enumerate}
  \item[a)] Demuestre que si $x$ es también invertible, entonces
    $$\{ x a_1, x x_2, \ldots, x x_{\phi (n)} \} = (\ZZ/n\ZZ)^\times.$$

  \item[b)] Use el punto anterior para probar la congruencia de Euler:
    $x^{\phi(n)} = 1$ para $x \in (\ZZ/n\ZZ)^\times$,
    o de manera equivalente:
    $$a^{\phi (n)} \equiv 1 \pmod{n} \quad \text{para }\mcd (a,n) = 1.$$
  \end{enumerate}
\end{problema}

La congruencia de Euler generaliza el pequeño teorema de Fermat
(problema~\ref{probl:pequeno-Fermat}).

\begin{problema}
  Demuestre las siguientes identidades para cualesquiera $m,n$:

  \begin{enumerate}
  \item[a)] $\phi (mn) = \phi (m)\,\phi (n) \, \frac{d}{\phi (d)}$,
    donde $d = \mcd (m,n)$.

  \item[b)] $\phi(\mcd (m,n)) \, \phi(\mcm (m,n)) = \phi(m)\,\phi(n)$.
  \end{enumerate}
\end{problema}

\begin{problema}[Gauss]
  Para $n \ge 1$ consideremos las fracciones
  \[
    \frac{1}{n}, ~
    \frac{2}{n}, ~
    \frac{3}{n}, ~
    \ldots, ~
    \frac{n-1}{n}, ~
    \frac{n}{n}.
  \]
  Luego escribamos cada una de la forma $\frac{a}{b}$ con $\mcd(a,b) = 1$.

  \begin{enumerate}
  \item[a)] Demuestre que para cada $d \mid n$, el número de fracciones en la
    lista con $d$ en el denominador es precisamente $\phi (d)$.

  \item[b)] Deduzca la identidad $\sum_{d\mid n} \phi (d) = n$, donde la suma es
    sobre todos los divisores de $n$.
  \end{enumerate}
\end{problema}

\begin{ejemplo}
  Para $n = 12$ tenemos
  \[ \phi (1) + \phi (2) + \phi (3) + \phi (4) + \phi (6) + \phi (12) =
    1 + 1 + 2 + 2 + 2 + 4 = 12. \]
\end{ejemplo}

\begin{problema}
  Demuestre que un primo $p$ satisface $\phi (p) = 2\,\phi (p-1)$ si y solamente
  si $p = 2^{2^k} + 1$ para algún $k$ (es decir, es un primo de Fermat).
\end{problema}

\begin{problema}
  Calcule la suma de los números coprimos con $n$:
  $$\sum_{\substack{1 \le t \le n \\ \mcd (t,n) = 1}} t = \frac{\phi(n)}{2}\,n.$$

  Sugerencia: escriba la suma de dos maneras:
  \[
    \sum_{\substack{1 \le t \le n \\ \gcd (t,n) = 1}} t =
    \sum_{\substack{1 \le n-t \le n \\ \gcd (n-t,n) = 1}} (n-t) =
    \sum_{\substack{1 \le t \le n \\ \gcd (t,n) = 1}} (n-t).
  \]
\end{problema}

%%%%%%%%%%%%%%%%%%%%%%%%%%%%%%%%%%%%%%%%%%%%%%%%%%%%%%%%%%%%%%%%%%%%%%%%%%%%%%%%

\section{Teorema de Lagrange sobre polinomios mód $p$}

Un resultado importante que me gustaría presentar es el siguiente teorema,
descubierto por Lagrange.

\begin{framed}
  Sea $p$ un número primo. Consideremos un polinomio con coeficientes enteros
  $$f (x) = a_n\,x^n + \cdots + a_1\,x + a_0,$$
  donde $p \nmid a_n$. Entonces, la congruencia $f (x) \equiv 0 \pmod{p}$ tiene
  $\le n$ soluciones.
\end{framed}

Recuerde que un polinomio de grado $n$ siempre tiene $\le n$ raíces
(reales o complejas). El resultado de arriba nos dice que lo mismo sucede con
congruencias polinomiales mód $p$.  El argumento es bastante sutil, así que
lo voy a dar por completo, sin convertirlo en otro problema más.

\begin{proof}[Demostración del teorema de Lagrange]
  Se procede por inducción sobre $n$. El caso base es $n = 1$, cuando el
  polinomio en cuestión es lineal.

  Para el paso inductivo, supongamos que el resultado se cumple para los grados
  $< n$. Ahora si $f (x) \equiv 0 \pmod{p}$ no tiene soluciones, no hay que
  probar nada. Sino, sea $a \in \ZZ$ una solución, así que
  $f (a) \equiv 0 \pmod{p}$. Podemos dividir $f (x)$ con resto por el polinomio
  mónico $x - a$. Nos quedará
  $$f (x) = (x-a)\,g(x) + r,$$
  donde
  $$g (x) = b_{n-1}\,x^{n-1} + b_{n-2}\,x^{n-2} + \cdots + b_1\,x + b_0,$$
  y $p \nmid b_{n-1} = a_n$. Sustituyendo $x = a$, notamos que
  $r = f (a)$. Entonces, módulo $p$ se tiene
  $$f (x) \equiv (x-a)\,g(x) \pmod{p}.$$
  Ahora dado que $p$ es primo (!), tenemos
  \[
    f (x) \equiv 0 \iff
    x \equiv a \text{ o }
    g (x) \equiv 0 \pmod{p}.
  \]
  Por la hipótesis de inducción, $g (x) \equiv 0 \pmod{p}$ tiene $\le n-1$
  soluciones.
\end{proof}

\begin{ejemplo}
  El teorema de Lagrange implica que la congruencia $x^n \equiv 1 \pmod{p}$
  no puede tener más de $n$ soluciones mód $p$. En particular,
  $x^2 \equiv 1 \pmod{p}$ tiene solamente las soluciones obvias
  $x \equiv \pm 1$. Sin embargo, hay $4$ soluciones mód $8$:
  $$1^2 \equiv 3^2 \equiv 5^2 \equiv 7^2 \equiv 1 \pmod{8}.$$
  Esto sucede porque $8$ es un número compuesto.
\end{ejemplo}

\begin{problema}[Teorema de Wilson-2]
  \label{probl:Wilson-2}
  Consideremos la congruencia $x^{p-1} \equiv 1 \pmod{p}$.

  \begin{enumerate}
  \item[a)] Combine el pequeño teorema de Fermat
    (problema~\ref{probl:pequeno-Fermat}) con nuestra prueba del teorema de
    Lagrange para establecer la congruencia de polinomios
    $$x^{p-1} - 1 \equiv (x - 1)\,(x - 2)\cdots (x - (p-1)) \pmod{p}.$$

  \item[b)] Deduzca de a) el teorema de Wilson:
    $(p - 1)! \equiv -1 \pmod{p}$.

  (Véase también \ref{probl:Wilson-1}.)
  \end{enumerate}
\end{problema}

\begin{problema}
  Use el teorema de Lagrange para probar que si $d \mid (p-1)$, entonces la
  congruencia $x^d \equiv 1 \pmod{p}$ tiene precisamente $d$ soluciones.

  Sugerencia: factorice $x^{p-1} - 1 = (x^d - 1)\,(\cdots)$, y recuerde
  el pequeño teorema de Fermat (problema~\ref{probl:pequeno-Fermat}).
\end{problema}

\begin{problema}
  Demuestre que $x^{p-2} + x^{p-1} + \cdots + x + 1 \equiv 0 \pmod{p}$
  tiene exactamente $p-2$ soluciones. (¿Cuáles son?)
\end{problema}

%%%%%%%%%%%%%%%%%%%%%%%%%%%%%%%%%%%%%%%%%%%%%%%%%%%%%%%%%%%%%%%%%%%%%%%%%%%%%%%%

\section{Orden multiplicativo mód $n$}

\begin{problema}
  Consideremos $x \in (\ZZ/n\ZZ)^\times$.

  \begin{enumerate}
  \item[a)] Demuestre que existen $k > \ell$ tales que $x^k = x^\ell$ en
    $(\ZZ/n\ZZ)^\times$.

  \item[b)] Concluya que $x^k = 1$ para algún $k = 1,2,3,\ldots$
  \end{enumerate}
\end{problema}

\begin{definicion}
  Sea $x \in (\ZZ/n\ZZ)^\times$ un resto invertible mód $n$. Su \textbf{orden}
  (\textbf{multiplicativo}) es el mínimo $k = 1,2,3,\ldots$ tal que
  $x^k = 1$.
\end{definicion}

\begin{ejemplo}
  Consideremos las potencias de $2$ módulo $5$:
  \[
    2^2 = 4, ~
    2^3 \equiv 3, ~
    2^4 \equiv 1 \pmod{5}.
  \]
  Entonces, $[2]_5$ tiene orden $4$. Por otra parte, módulo $7$ tenemos
  \[
    2^2 = 4, ~
    2^3 = 8 \equiv 1 \pmod{7}.
  \]
  Esto significa que el orden de $[2]_7$ es $2$.
\end{ejemplo}

\begin{problema}
  Para $x \in (\ZZ/n\ZZ)^\times$, sea $k = \ord (x)$.

  \begin{enumerate}
  \item[a)] Si $\ell = qk + r$, demuestre que $x^\ell = 1$ si y solamente si
    $x^r = 1$.

  \item[b)] Deduzca que $x^\ell = 1$ si y solamente si $\ell \mid k$.

    (Use la división con resto.)

  \item[c)] Demuestre que $\ord (x) \mid \phi(n)$ para todo
    $x \in (\ZZ/n\ZZ)^\times$.

    (Recuerde la congruencia de Euler \ref{probl:congruencia-de-euler}.)

  \item[d)] Demuestre que $x^\ell = x^m$ si y solamente si
    $\ell \equiv m \pmod{k}$.
  \end{enumerate}
\end{problema}

\begin{problema}
  Para $x \in (\ZZ/n\ZZ)^\times$ y $\ell = 1,2,3,\ldots$ deduzca la fórmula
  $$\ord (x^\ell) = \frac{\ord (x)}{\mcd (\ord (x), \ell)}.$$
\end{problema}

\begin{problema}
  Consideremos $x,y \in (\ZZ/n\ZZ)^\times$.

  \begin{itemize}
  \item[a)] Demuestre que $\ord (xy) = \ord(x) \, \ord(y)$ si
    $\mcd (\ord(x), \ord(y)) = 1$.
  \item[b)] ¿Qué pasa si $\mcd (\ord(x), \ord(y)) \ne 1$?
  \end{itemize}
\end{problema}

%%%%%%%%%%%%%%%%%%%%%%%%%%%%%%%%%%%%%%%%%%%%%%%%%%%%%%%%%%%%%%%%%%%%%%%%%%%%%%%%

\section{Raíces primitivas mód $p$}

Un resultado importante sobre los restos módulo $p$ es el siguiente.

\begin{framed}
  Para todo primo $p$ existe un elemento $x \in (\ZZ/p\ZZ)^\times$ con
  $\ord (x) = p-1 = \phi(p) = \# (\ZZ/p\ZZ)^\times$. En otras palabras,
  $$(\ZZ/p\ZZ)^\times = \{ 1, x, x^2, \ldots, x^{p-2} \}.$$
  Este $x$ se llama una \textbf{raíz primitiva} mód $p$.
\end{framed}

\begin{ejemplo}
  Para $p = 13$ como una raíz primitiva funciona $x = [2]_{13}$:
  \[
    2^0 \equiv 1, ~
    2^1 \equiv 2, ~
    2^2 \equiv 4, ~
    2^3 \equiv 8, ~
    2^4 \equiv 3, ~
    2^5 \equiv 6, ~
    2^6 \equiv 12, ~
    2^7 \equiv 11, ~
    2^8 \equiv 9, ~
    2^9 \equiv 5, ~
    2^{10} \equiv 10, ~
    2^{11} \equiv 7.
  \]
\end{ejemplo}

\begin{ejemplo}
  Módulo $15$ hay $8$ elementos invertibles, y sus ordenes son los siguientes:
  \begin{center}
    \begin{tabular}{rcccccccc}
      \hline
      $a$ & $1$ & $2$ & $4$ & $7$ & $8$ & $11$ & $13$ & $14$ \\
      \hline
      $\ord [a]_{15}$ & $1$ & $4$ & $2$ & $4$ & $4$ & $2$ & $4$ & $2$ \\
      \hline
    \end{tabular}
  \end{center}
  Entonces, no hay elemento $x$ tal que
  $(\ZZ/15\ZZ)^\times = \{ 1, x, x^2, \ldots, x^7 \}$.
  Esto sucede porque $15$ es compuesto.
\end{ejemplo}

En estas notas \textbf{no} vamos a probar la existencia de raíz primitiva.
No es muy difícil, pero el argumento es un poco técnico y nos llevaría un poco
lejos. Lo que pasa es que la prueba no es constructiva, y en general no existe
una fórmula sencilla que para un primo $p$ dé una raíz primitiva mód $p$.
He aquí una pequeña lista de raíces primitivas módulo los primeros diez primos:

\begin{align*}
  p = 2\colon & 1, \\
  p = 3\colon & 2, \\
  p = 5\colon & 2, 3, \\
  p = 7\colon & 3, 5, \\
  p = 11\colon & 2, 6, 7, 8, \\
  p = 13\colon & 2, 6, 7, 11, \\
  p = 17\colon & 3, 5, 6, 7, 10, 11, 12, 14, \\
  p = 19\colon & 2, 3, 10, 13, 14, 15, \\
  p = 23\colon & 5, 7, 10, 11, 14, 15, 17, 19, 20, 21, \\
  p = 29\colon & 2, 3, 8, 10, 11, 14, 15, 18, 19, 21, 26, 27, \\
              & \cdots
\end{align*}

En el resto de problemas, $p$ es un número primo, y se puede asumir existencia
de una raíz primitiva $x \in (\ZZ/p\ZZ)^\times$.

\begin{problema}
  Demuestre que para cada $d \mid (p-1)$, en $(\ZZ/p\ZZ)^\times$ hay exactamente
  $\phi(d)$ elementos de orden $d$.
\end{problema}

\begin{comentario}
  En particular, el problema anterior nos dice que hay $\phi (p-1)$ diferentes
  raíces primitivas mód $p$. El número $\phi (p-1)$ no es tan pequeño respecto a
  $p-1$, así que en práctica, para encontrar una raíz primitiva mód $p$,
  se puede escoger un número $1 < a < p-1$ al azar, y luego comprobar si
  $\ord [a]_p = p-1$.
\end{comentario}

\begin{ejemplo}
  He aquí los ordenes de los restos módulo $p = 13$:

  \begin{center}
    \begin{tabular}{rccccccccccccc}
      \hline
      $a$: & $1$ & $2$ & $3$ & $4$ & $5$ & $6$ & $7$ & $8$ & $9$ & $10$ & $11$ & $12$ \\
      \hline
      $\ord [a]_{13}$: & $1$ & $12$ & $3$ & $6$ & $4$ & $12$ & $12$ & $4$ & $3$ & $6$ & $12$ & $2$ \\
      \hline
    \end{tabular}
  \end{center}
\end{ejemplo}

\begin{problema}
  ~

  \begin{enumerate}
  \item[a)] Demuestre que si $x$, $x'$ son dos raíces primitivas mód $p$,
    entonces $x x'$ no es una raíz primitiva mód $p$.

  \item[b)] Demuestre que si $x$ es una raíz primitiva mód $p$, entonces
    $x^{-1}$ es también una raíz primitiva mód $p$.
  \end{enumerate}
\end{problema}

\begin{problema}
  Sea $p$ un primo impar. Demuestre que existe $a \in \ZZ$ tal que
  $a^2 \equiv -1 \pmod{p}$ si y solamente si $p \equiv 1 \pmod{4}$.
\end{problema}

\begin{problema}[Euler]
  Sea $p$ un primo impar. Demuestre que para $p \nmid a$ se tiene
  la congruencia mód $p$
  \[ a^{\frac{p-1}{2}} \equiv
    \begin{cases}
      +1, & a \text{ es cuadrado mód }p, \\
      -1, & a \text{ no es cuadrado mód }p.
    \end{cases} \]
\end{problema}

\begin{problema}
  Investigue para cuáles primos $p$ existe $x \in \ZZ/p\ZZ$, tal que $x^3 = 1$ y
  $x \ne 0$.
\end{problema}

\begin{problema}
  Sea $p$ un número primo.

  \begin{enumerate}
  \item[a)] Si $p \equiv 1 \pmod{4}$, demuestre que $a$ es una raíz primitiva
    módulo $p$ si y solamente si $-a$ lo es.

  \item[b)] Si $p \equiv 3 \pmod{4}$, demuestre que $a$ es una raíz primitiva
    módulo $p$ si y solamente si $-a$ tiene orden $\frac{p-1}{2}$.
  \end{enumerate}
\end{problema}

\begin{problema}
  Demuestre que
  $$1^k + 2^k + \cdots + (p-1)^k \equiv 0 \pmod{p}$$
  si $p-1 \nmid k$. Por ejemplo,
  $$1^3 + 2^3 + 3^3 + 4^3 = 100 \equiv 0 \pmod{5}.$$
\end{problema}

\begin{problema}[Gauss]
  Demuestre que si $a_1, \ldots, a_s$ son diferentes raíces primitivas módulo
  $p$, entonces
  $$a_1 \cdots a_s \equiv 1 \pmod{p}.$$
\end{problema}

\begin{problema}[Teorema de Wilson-3]
  Use la existencia de raíz para probar que $(p-1)! \equiv -1 \pmod{p}$.

  (Véase también \ref{probl:Wilson-1} y \ref{probl:Wilson-2}.)
\end{problema}

\begin{problema}
  Sea $p$ un primo impar.

  \begin{enumerate}
  \item[a)] Demuestre que en $(\ZZ/p\ZZ)^\times$ hay exactamente $\frac{p-1}{2}$
    cuadrados (elementos de la forma $x^2$ para $x \in (\ZZ/p\ZZ)^\times$).

  \item[b)] Demuestre que los conjuntos $X = \{ x^2 \mid x \in \ZZ/p\ZZ \}$ e
    $Y = \{ -1-y^2 \mid y \in \ZZ/p\ZZ \}$ tienen intersección no vacía.

  \item[c)] Deduzca que siempre existen $m,n \in \ZZ$ tales que
    $m^2 + n^2 + 1 \equiv 0 \pmod{p}$.
  \end{enumerate}
\end{problema}

\end{document}
