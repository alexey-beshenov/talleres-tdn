\ifdefined\handout
  \documentclass[handout]{beamer}
\else
  \documentclass{beamer}
\fi

\usetheme{boxes}
\definecolor{beamer@structure@color}{rgb}{0,0,0}

\usecolortheme{structure}

\usepackage{framed}

\setbeamertemplate{footline}[frame number]
\setbeamertemplate{frametitle}{\color{black}
\def\myhrulefill{\leavevmode\leaders\hrule height 1pt\hfill\kern 0pt}
\headingfont\insertframetitle\par\vskip-8pt\myhrulefill}

\usepackage[spanish]{babel}

\DeclareMathOperator{\mcd}{mcd}

\setbeamertemplate{navigation symbols}{}

\definecolor{shadecolor}{rgb}{0.89,0.89,0.70}

\usepackage{mathspec}
\setsansfont[BoldFont={IBM Plex Sans Bold}, ItalicFont={IBM Plex Sans Italic}]{IBM Plex Sans}
\setmathrm[BoldFont={IBM Plex Sans Bold}, ItalicFont={IBM Plex Sans Italic}]{IBM Plex Sans}
\newfontfamily\headingfont[]{IBM Plex Sans Bold}

\ifdefined\solutions
\setbeamercovered{transparent=25}
\fi

\begin{document}

%%%%%%%%%%%%%%%%%%%%%%%%%%%%%%%%%%%%%%%%%%%%%%%%%%%%%%%%%%%%%%%%%%%%%%%%%%%%%%%%

\begin{frame}[noframenumbering]
  \begin{center}
    {\LARGE\bf Ejercicios de la teoría de números

    }

    \vspace{3em}

    {\large\bf Alexey Beshenov}

    \vspace{4em}

    09/11/2021

  \end{center}
\end{frame}

%%%%%%%%%%%%%%%%%%%%%%%%%%%%%%%%%%%%%%%%%%%%%%%%%%%%%%%%%%%%%%%%%%%%%%%%%%%%%%%%

\begin{frame}[fragile]
  \frametitle{Recordatorio de la vez pasada (!)}

  \begin{shaded}
    \textbf{IWYMIC 2002}: Encuentre el número de las soluciones enteras de
    $$\frac{1}{x} + \frac{1}{y} = \frac{1}{14}.$$
    Dé alguna solución particular.
  \end{shaded}

  \ifdefined\solutions
  \onslide<2->{\rule{2cm}{1pt}}

  \begin{itemize}
  \item<3-> La ecuación es equivalente a
    $$(x - 14)\,(y - 14) = 14^2, \quad (x,y) \ne (0,0).$$
    Por ejemplo, $x = 15$, $y = 14^2 + 14 = 210$ es una solución.

  \item<4-> Tenemos
    $$d (14^2) = d (2^2) \cdot d (7^2) = 3\cdot 3 = 9.$$
    Con signos $-1$, habrá también $9$ divisores.

  \item<5-> Quitando el caso de $(x,y) = (0,0)$, quedan $17$ soluciones.
  \end{itemize}
  \else\fi
\end{frame}

%%%%%%%%%%%%%%%%%%%%%%%%%%%%%%%%%%%%%%%%%%%%%%%%%%%%%%%%%%%%%%%%%%%%%%%%%%%%%%%%

\begin{frame}[fragile]
  \frametitle{Cuadrados}

  \begin{shaded}
    \textbf{IWIMIC 2005}: Encuentre todos los enteros $x$ tales que $x$ e $x+45$
    son cuadrados.
  \end{shaded}

  \ifdefined\solutions
  \onslide<2->{\rule{2cm}{1pt}}

  \begin{itemize}
  \item<3-> Escribiendo $x = y^2$, $y^2 + 45 = z^2$,
  \[ (y + z)\,(y - z) = -45 = -3^2\cdot 5, \]
  \[
    0 < y < z, \quad
    |y-z| < y + z.
  \]

  \item<4-> Tres casos:
  \[ (y+z, y-z) = (9,-5), ~ (15,-3), ~ (45,-1). \]
  Entonces, $(y,z) = (2,7), ~ (6,9), ~ (22,23)$.

  \item<5-> Conclusión: $x = 4, 36, 484$.
  \end{itemize}
  \fi
\end{frame}

%%%%%%%%%%%%%%%%%%%%%%%%%%%%%%%%%%%%%%%%%%%%%%%%%%%%%%%%%%%%%%%%%%%%%%%%%%%%%%%%

\begin{frame}[fragile]
  \frametitle{Divisibilidad por un número compuesto}

  \begin{shaded}
    Demuestre que el número
    $$f(n) = 23^n + 12^n - 32^n - 3^n$$
    es divisible por $35$ para todo $n \ge 1$ impar.
  \end{shaded}

  \ifdefined\solutions
  \onslide<2->{\rule{2cm}{1pt}}

  \begin{itemize}
  \item<3-> $f (n) \equiv 3^n + 2^n - 2^n - 3^n \pmod{5}$

  \item<4-> $f (n) \equiv 2^n + 5^n - 4^n - 3^n$

    \hspace{1.9em}$\equiv 2^n + (-2)^n - (-3)^n - 3^n \pmod{7}$
  \end{itemize}
  \fi
\end{frame}

%%%%%%%%%%%%%%%%%%%%%%%%%%%%%%%%%%%%%%%%%%%%%%%%%%%%%%%%%%%%%%%%%%%%%%%%%%%%%%%%

\begin{frame}
  \frametitle{Tarea}

  \begin{shaded}
    \textbf{IWYMIC 2006}: Demuestre que el número
    $$1596^n + 1000^n - 270^n - 320^n$$
    es divisible por $2006$ para todo $n \ge 1$ impar.
  \end{shaded}
\end{frame}

%%%%%%%%%%%%%%%%%%%%%%%%%%%%%%%%%%%%%%%%%%%%%%%%%%%%%%%%%%%%%%%%%%%%%%%%%%%%%%%%

\begin{frame}[fragile]
  \frametitle{Contando potencias}

  \begin{shaded}
    \begin{itemize}
    \item ¿Cuántos números enteros $x \le N$ son $k$-ésimas potencias?

    \item ¿Cuántos números $x \le N$ no son cuadrados, ni cubos?
    \end{itemize}
  \end{shaded}

  \vspace{\fill}

  \ifdefined\solutions
  \onslide<2->{\rule{2cm}{1pt}}

  \begin{itemize}
  \item<3-> $\lfloor \sqrt[k]{N}\rfloor$ números $1, 2^k, 3^k, 4^k, \ldots$

  \item<4-> Inclusión-exclusión:
    $f(N) = N - \lfloor\sqrt{N}\rfloor - \lfloor\sqrt[3]{N}\rfloor + \lfloor\sqrt[6]{N}\rfloor$.
  \end{itemize}
  \fi
\end{frame}

%%%%%%%%%%%%%%%%%%%%%%%%%%%%%%%%%%%%%%%%%%%%%%%%%%%%%%%%%%%%%%%%%%%%%%%%%%%%%%%%

\begin{frame}[fragile]
  \frametitle{Contando potencias (cont.)}

  \begin{shaded}
    \textbf{IWYMIC 2005}: Consideremos la sucesión de los números enteros que no
    son cuadrados ni cubos:
    $$2, 3, 5, 6, 7, 10, 11, 12, 13, 14, \ldots$$
    Encuentre el término número $1000$.
  \end{shaded}

  \ifdefined\solutions
  \onslide<2->{\rule{2cm}{1pt}}

  \begin{itemize}
  \item<3-> Buscamos $f (N) = 1000$ para
    $f(N) = N - \lfloor\sqrt{N}\rfloor - \lfloor\sqrt[3]{N}\rfloor + \lfloor\sqrt[6]{N}\rfloor$.

  \item<4-> $f (1000) = 1000 - 31 - 10 + 3 = 962$

  \item<5-> $f (1050) = 1050 - 32 - 10 + 3 = 1011$

  \item<6-> $f (1039) = 1039 - 32 - 10 + 3 = 1000$

    (prueba y error)
  \end{itemize}
  \fi
\end{frame}

%%%%%%%%%%%%%%%%%%%%%%%%%%%%%%%%%%%%%%%%%%%%%%%%%%%%%%%%%%%%%%%%%%%%%%%%%%%%%%%%

\begin{frame}
  \frametitle{Tarea}

  \begin{shaded}
    \begin{itemize}
    \item ¿Cuántos números $x \le N$ no son cuadrados, ni cubos,
      \underline{ni quintas potencias}?

    \item ¿Cuál es el término número $4321$ en la sucesión correspondiente?
    \end{itemize}
  \end{shaded}
\end{frame}

%%%%%%%%%%%%%%%%%%%%%%%%%%%%%%%%%%%%%%%%%%%%%%%%%%%%%%%%%%%%%%%%%%%%%%%%%%%%%%%%

\begin{frame}[fragile]
  \frametitle{Sumas de dos cuadrados}

  \begin{shaded}
    Resuelve la congruencia
    $x^2 \equiv -1 \pmod{p}$
    para $p = 5, 7, 11, 13$.
  \end{shaded}

  \ifdefined\solutions
  \onslide<2->{\rule{2cm}{1pt}}

  \begin{itemize}
  \item<3-> $(\pm 2)^2 = 4 \equiv -1 \pmod{5}$.

  \item<4-> $(\pm 1)^2 = 1$, $(\pm 2)^2 = 4$, $(\pm 3)^2 \equiv 2$

    son todos los cuadrados mód $7$.

    Otra opción: usando una raíz primitiva $3$:
    \[
      \underline{3^0 \equiv 1}, \,
      3, \,
      \underline{3^2 \equiv 2}, \,
      3^3 \equiv 6, \,
      \underline{3^4 \equiv 4}, \,
      3^5 \equiv 5.
    \]

  \item<5->
    $(\pm 1)^2 = 1$,
    $(\pm 2)^2 = 4$,
    $(\pm 3)^2 = 9$,
    $(\pm 4)^2 \equiv 5$,
    $(\pm 5)^2 \equiv 3$

    son todos los cuadrados mód $11$.


  \item<6-> $(\pm 5)^2 \equiv -1 \pmod{13}$.
  \end{itemize}
  \fi
\end{frame}

%%%%%%%%%%%%%%%%%%%%%%%%%%%%%%%%%%%%%%%%%%%%%%%%%%%%%%%%%%%%%%%%%%%%%%%%%%%%%%%%

\begin{frame}[fragile]
  \frametitle{Sumas de dos cuadrados (cont.)}

  \begin{shaded}
    Para un primo impar $p$, demuestre que $x^2 \equiv -1 \pmod{p}$ tiene
    soluciones si y solo si $p \equiv 1 \pmod{4}$.
  \end{shaded}

  * Sugerencia: \emph{existe una \underline{raíz primitiva} $a$ tal que
    $a, a^2, \ldots, a^{p-2}, a^{p-1} \equiv 1$ son todos los residuos no nulos mód
    $p$}.

  \vspace{\fill}
  
  \ifdefined\solutions
  \onslide<2->{\rule{2cm}{1pt}}

  \begin{itemize}
  \item<3-> $x^2 \equiv -1$, $x^3 \equiv -x \not\equiv 1$, $x^4 \equiv 1$.

  \item<4-> Pequeño teorema de Fermat: $x^{p-1} \equiv 1$.

  \item<5-> $p-1 = 4n + r$, donde $0 \le r < 4$.

  \item<6-> $1 = x^{p-1} = (x^4)^n\cdot x^r = x^r$, entonces $r = 0$.

  \item<7-> Si $4 \mid p-1$, funciona $x = a^{\frac{p-1}{4}}$.
  \end{itemize}
  \fi
\end{frame}

%%%%%%%%%%%%%%%%%%%%%%%%%%%%%%%%%%%%%%%%%%%%%%%%%%%%%%%%%%%%%%%%%%%%%%%%%%%%%%%%

\begin{frame}[fragile]
  \frametitle{Sumas de dos cuadrados (cont.) / Tarea}

  \begin{shaded}
    Sea $p$ un primo impar. Demuestre que $x^2 + y^2 \equiv 0 \pmod{p}$ tiene
    soluciones no triviales $(x,y) \not\equiv (0,0) \pmod{p}$ si y solo si
    $p \equiv 1 \pmod{4}$.
  \end{shaded}

  \vspace{\fill}

  \iffalse
  \ifdefined\solutions
  \onslide<2->{\rule{2cm}{1pt}}

  \begin{itemize}
  \item<3-> Recordatorio: si $y \not\equiv 0$, entonces existe $y^{-1}$ tal que
    $y y^{-1} \equiv 1 \pmod{p}$.

  \item<4-> $x^2 + y^2 \equiv 0$, $y \not\equiv 0$ $\Longrightarrow$
    $(x\,y^{-1})^2 \equiv -1$.
  \end{itemize}
  \fi
  \fi
\end{frame}

%%%%%%%%%%%%%%%%%%%%%%%%%%%%%%%%%%%%%%%%%%%%%%%%%%%%%%%%%%%%%%%%%%%%%%%%%%%%%%%%

\begin{frame}[fragile]
  \frametitle{Sumas de dos cuadrados (cont.) / Tarea}

  \begin{shaded}
    \textbf{IWYMIC 2004}: ¿Cuántas soluciones enteras tiene la ecuación
    $x^2 + y^2 - 16y = 2004$?
  \end{shaded}

  \vspace{\fill}

  \iffalse
  \ifdefined\solutions
  \onslide<2->{\rule{2cm}{1pt}}

  \begin{itemize}
  \item<3-> $x^2 + (y - 8)^2 = 2068 = 4\cdot 11\cdot 47$.

  \item<4-> $11 \equiv 47 \equiv 3 \pmod{4}$.

  \item<5-> Conclusión: no hay soluciones.
  \end{itemize}
  \fi
  \fi
\end{frame}

%%%%%%%%%%%%%%%%%%%%%%%%%%%%%%%%%%%%%%%%%%%%%%%%%%%%%%%%%%%%%%%%%%%%%%%%%%%%%%%%

\begin{frame}
  \frametitle{Tarea}

  \begin{shaded}
    Verifique si $x^2 + y^2 - 16y = 2020$ tiene soluciones.

    * Difícil: ¿Cuántas son en total?
  \end{shaded}

  \vspace{1em}

  \textbf{Lectura adicional}: <<El Libro de las Demostraciones>>, cap. 4,
  Representación de enteros como suma de dos cuadrados.
\end{frame}

%%%%%%%%%%%%%%%%%%%%%%%%%%%%%%%%%%%%%%%%%%%%%%%%%%%%%%%%%%%%%%%%%%%%%%%%%%%%%%%%

\end{document}
