\documentclass{article}

\usepackage{fullpage}

\usepackage[utf8]{inputenc}
\usepackage[spanish]{babel}

\usepackage[T1]{fontenc}
\usepackage{fourier}
\usepackage{paratype}

\usepackage{framed}
\usepackage{hyperref}
\usepackage{amsmath,amssymb}
\usepackage{tikz}

\newcommand{\ZZ}{\mathbb{Z}}
\DeclareMathOperator{\mcd}{mcd}
\DeclareMathOperator{\mcm}{mcm}
\DeclareMathOperator{\ord}{ord}

\usepackage{amsthm}
\theoremstyle{definition}
\newtheorem{problema}{Problema}[section]
\newtheorem*{comentario}{Comentario}
\newtheorem*{definicion}{Definición}
\newtheorem*{ejemplo}{Ejemplo}

\newenvironment{solucion}{\begin{proof}[Solución]\small}{\end{proof}}

\renewcommand*{\thefootnote}{\fnsymbol{footnote}}


\title{Hoja 4: Teorema de Lagrange sobre polinomios mód $p$}
\author{Alexey Beshenov (cadadr@gmail.com)}
\date{23 de septiembre de 2021}

\begin{document}

\maketitle

\setcounter{section}{4}

Un resultado importante que me gustaría presentar a continuación es el siguiente
teorema, descubierto por Lagrange.

\begin{framed}
  Sea $p$ un número primo. Consideremos un polinomio con coeficientes enteros
  $$f (x) = a_n\,x^n + \cdots + a_1\,x + a_0,$$
  donde $p \nmid a_n$. Entonces, la congruencia $f (x) \equiv 0 \pmod{p}$ tiene
  $\le n$ soluciones.
\end{framed}

Recuerde que un polinomio de grado $n$ siempre tiene $\le n$ raíces
reales o complejas. El resultado de arriba nos dice que lo mismo sucede con
congruencias polinomiales mód $p$. El argumento es bastante sutil, así que
lo voy a dar por completo, sin convertirlo en otro problema más.

\begin{proof}[Demostración del teorema de Lagrange]
  Se procede por inducción sobre $n$. El caso base es $n = 1$, cuando el
  polinomio en cuestión es lineal.

  Para el paso inductivo, supongamos que el resultado se cumple para los grados
  $< n$. Ahora si $f (x) \equiv 0 \pmod{p}$ no tiene soluciones, no hay que
  probar nada. Sino, sea $a \in \ZZ$ una solución, así que
  $f (a) \equiv 0 \pmod{p}$. Podemos dividir $f (x)$ por el polinomio mónico
  lineal $x - a$. Nos quedará
  $$f (x) = (x-a)\,g(x) + r,$$
  donde
  $$g (x) = b_{n-1}\,x^{n-1} + b_{n-2}\,x^{n-2} + \cdots + b_1\,x + b_0,$$
  y $p \nmid b_{n-1} = a_n$. Sustituyendo $x = a$, notamos que
  $r = f (a)$. Entonces, módulo $p$ se tiene
  $$f (x) \equiv (x-a)\,g(x) \pmod{p}.$$
  Ahora dado que $p$ es primo (!), tenemos
  \[
    f (x) \equiv 0 \iff
    x \equiv a \text{ o }
    g (x) \equiv 0 \pmod{p}.
  \]
  Por la hipótesis de inducción, $g (x) \equiv 0 \pmod{p}$ tiene $\le n-1$
  soluciones.
\end{proof}

\begin{ejemplo}
  El teorema de Lagrange implica que la congruencia $x^n \equiv 1 \pmod{p}$
  no puede tener más de $n$ soluciones mód $p$. En particular,
  $x^2 \equiv 1 \pmod{p}$ tiene solamente las soluciones obvias
  $x \equiv \pm 1$. Sin embargo, hay $4$ soluciones mód $8$:
  $$1^2 \equiv 3^2 \equiv 5^2 \equiv 7^2 \equiv 1 \pmod{8}.$$
  Esto sucede porque $8$ es un número compuesto.
\end{ejemplo}

\pagebreak

\begin{problema}[Teorema de Wilson-2]
  \label{probl:Wilson-2}
  Consideremos la congruencia $x^{p-1} \equiv 1 \pmod{p}$.

  \begin{enumerate}
  \item[a)] Combine el pequeño teorema de Fermat con nuestra prueba del teorema
    de Lagrange para establecer la congruencia de polinomios
    $$x^{p-1} - 1 \equiv (x - 1)\,(x - 2)\cdots (x - (p-1)) \pmod{p}.$$

  \item[b)] Deduzca de a) el teorema de Wilson:
    $(p - 1)! \equiv -1 \pmod{p}$.
  \end{enumerate}
\end{problema}

\begin{problema}
  Use el teorema de Lagrange para probar que si $d \mid (p-1)$, entonces la
  congruencia $x^d \equiv 1 \pmod{p}$ tiene precisamente $d$ soluciones.

  Sugerencia: factorice $x^{p-1} - 1 = (x^d - 1)\,(\cdots)$, y recuerde
  el pequeño teorema de Fermat.
\end{problema}

\begin{problema}
  Demuestre que $x^{p-2} + x^{p-1} + \cdots + x + 1 \equiv 0 \pmod{p}$
  tiene exactamente $p-2$ soluciones. ¿Cuáles son?
\end{problema}

\end{document}
